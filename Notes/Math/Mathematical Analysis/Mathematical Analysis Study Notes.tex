\documentclass[12pt]{article} 

\usepackage{amsmath, amssymb, amsthm}
\usepackage{enumitem}
\usepackage{titlesec}

\titleformat{\section}[block]{\normalfont\LARGE\bfseries\centering}{\thesection}{1em}{}
\titleformat{\subsection}[runin]{\normalfont\Large\bfseries}{\thesubsection}{1em}{}

\usepackage[a4paper, top=2.5cm, bottom=2.5cm, left=3cm, right=3cm]{geometry}

\usepackage{setspace}
\setstretch{1.25}              
\setlength{\parskip}{5pt}     
\setlength{\parindent}{0em} 

\usepackage{fancyhdr}
\setlength{\headheight}{16pt}
\pagestyle{fancy}
\fancyhf{}
\fancyhead[C]{\nouppercase{\leftmark}} 
\fancyfoot[C]{\thepage}                
\renewcommand{\headrulewidth}{0.4pt}
\renewcommand{\footrulewidth}{0pt}

\newtheoremstyle{oneline}% name
  {}% space above
  {}% space below
  {\normalfont}% body font
  {}% indent amount
  {\bfseries}% theorem head font
  {}% punctuation after theorem head
  {0.6em}% space after theorem head
  {\thmname{#1}\thmnumber{ #2}\thmnote{\textit{ (#3)}}}
\theoremstyle{oneline}

\newtheorem{theorem}{Theorem}[subsection]
\newtheorem{lemma}{Lemma}[subsection]
\newtheorem{remark}{Remark}[subsection]
\newtheorem{proposition}{Proposition}[subsection]
\newtheorem{corollary}{Corollary}[subsection]
\newtheorem{definition}{Definition}[subsection]
\newtheorem{example}{Example}[subsection]
\newtheorem{property}{Property}[subsection]

\newtheoremstyle{twoline}% name
  {1em}% space above
  {0.5em}% empace below
  {\rmfamily}% body font
  {}% indent amount
  {\bfseries}% theorem head font
  {}% punctuation after theorem head
  {\newline}% space after theorem head
  {\large\thmname{#1}}
\theoremstyle{twoline}
\newtheorem{introduction}{Introduction}
\newtheorem{exercise}{Exercise}[section]

\makeatletter 
\@addtoreset{equation}{section}
\makeatother
\renewcommand{\theequation}{\thesection.\arabic{equation}}


\title{Mathematical Analysis Study Notes}
\author{Emrys}
\date{}

\begin{document} 
\maketitle
\tableofcontents
\newpage

\section{Numerical Series}

\subsection{Series}
\begin{definition}
  Given a sequence $\{a_n\}$,we use the notation
   \[\sum\limits_{n=p}^{q}a_n\quad (p\leq q)\]
  to denote the sum $a_p+a_{p+1}+\cdots+a_q$. And we call the symbolic expression
  \[\sum\limits_{n=1}^{\infty}a_n=a_1+a_2+\cdots+a_n+\cdots\]
  an infinite series, or just a series.

  With $\{a_n\}$ we associate a sequence $\{S_n\}$, where
  \[S_n=\sum\limits_{k=1}^n a_k.\]
  The numbers $S_n$ are called the partial sums of the series.

  If $\{S_n\}$ converges to $s$, we say that the series converges, and write
  \[\sum\limits_{n=1}^{\infty}a_n=s.\]
  The number $s$ is called the sum of seires; If $\{S_n\}$ diverges, the series is said to diverge.
\end{definition}

\begin{remark}
  The above $s$ is called the sum of the series; but it should be clearly understood that s is the limit of a sequence of sums, and is not obtained simply by addition.
\end{remark}

\begin{remark}
  It is clear that every theorem about sequences can be state in series. The Cauchy criterion can be restated in the following form. 
\end{remark}

\begin{remark}
  For convenience of notation, we shall simply write $\Sigma a_n$ in place of $\sum\limits_{n=1}^{\infty}$.
\end{remark}

\begin{theorem}[Cauchy criterion]
  $\Sigma a_n$ converges $\iff$ for every $\varepsilon>0$ there exists an integer $N$ such that if $m\geq n\geq M$
  \begin{align}
    |\sum\limits_{k=n}^m a_k|\leq \varepsilon
  \end{align}
  It can be poved by the Cauchy criterion of the limit of sequences.
  And In particular, by taking $m=n$, $(\thesection.1)$ becomes
  \[|a_n|\leq\varepsilon\quad(n\geq N).\]
  So we can get the following vanishing condition.
\end{theorem}

\begin{proposition}[Absolute Value Property]
  $\Sigma a_n$ converges if $\Sigma |a_n|$ converges.\\
  \textit{Proof:} Since
  \[|a_{m+1}|+\cdots+a_n\leq||a_{m+1}|+\cdots+|a_n||,\]
  we can easily prove the original proposition by Cauchy criterion.
\end{proposition}

\begin{theorem}[Vanishing condition]
  If $\Sigma a_n$ converges, then $\lim\limits_{n\to\infty}a_n=0$.
\end{theorem}

\begin{remark}
  The condition $a_n\to0$ is not, however, sufficient to ensure convergence of $\Sigma a_n$. For instance, the series 
  \[\sum\limits_{n=1}^\infty \frac1n\]
  diverges;and we will prove it in the \textit{Example \thesubsection.2}.
\end{remark}

\begin{example}
  let $q\in\mathbb{R}$, discuss the convergence and divergence of the geometric series $\Sigma q^n$.\\
\textit{Solution:} We know that the partial sum of it is $S_n=\frac{q(1-q^n)}{1-q}$, so we have the series converges when $|q|<1$ and diverges when $|q|\geq1$.
\end{example}

\begin{example}
  Determine the convergence and divergence of the harmonic series $\Sigma \frac1n$.\\
  \textit{Solution:} When $n\geq 1$, we have
  \[\sum\limits_{k=n+1}^{2n}\frac1k\geq\sum\limits_{k=n+1}^{2n}\frac1{2n}=\frac12,\]
  so by Cauchy criterion, we know the original series diverges.
\end{example}

\subsection{Positive Series}
\introduction
\hspace*{2em}
When studying a subject, we usually proceed from the specific to the general. Therefore, when researching series, we first focus on a specific type of series. It is called the positive series, which, as the name suggests, are series where each term is a positive number. But next, we will also reseach the non-negetive series meanwhile.

\begin{theorem}[Fundamental Test]
  The positive series converges $\iff$ $\{S_n\}$ is bounded above.

\textit{Proof:} Because $a_n>0$, We conclude that this partial sum $S_n$ is monotonically increasing with respect to $n$. It follows from the fact that a monotonically increasing and bounded above sequence must converge that $\{S_n\}$ converges.
\end{theorem}

\begin{remark}
  The above test also holds for $a_n\geq0$.
\end{remark}

\begin{example}
  If $\Sigma a_n$ is a psoitive series, and $\frac{a_{n+1}}{a_n}<\frac{n}{n+\alpha}$, where $\alpha>1$, prove that $\Sigma a_n$ converges.\\
\textit{Proof:} Since
\[\frac{a_{n+1}}{a_n}<\frac{n}{n+\alpha},\]
it follows that 
\[a_n{n+1}<\frac1{\alpha-1(na_n-(n+1)a_{n+1}},\] 
Thus, we can conclude
\[S_n<a_1+\frac1{\alpha-1}\sum\limits_{k=1}^{n-1}(ka_k-(k+1)a_{k+1})=a_1+\frac1{\alpha-1}(a_1-na_n)<\frac{\alpha}{\alpha-1}a_1\] 
Then we have the series is bounded above, and by the above theorem, it follows that the seires converges.
\end{example}

\begin{theorem}[Cauchy Condensation Test]
  Suppose $a_1\geq a_2\geq a_3\geq\cdots\geq0$. Then the series $\sum\limits_{n=1}^{\infty} a_n$ converges if and only if the series $\sum\limits_{k=0}^\infty 2^ka_{2^k}$ converges.

\textit{Proof:} By the \textit{Theorem \thesubsection.1}, it suffices to consider boundness of the partial sums. let
\begin{align*}
  S_n=\sum\limits_{k=1}^n&=a_1+a_2+\cdots+a_n,\\
  T_m=\sum\limits_{k=0}^m2^ka^{2^k}&=a_1+2a_2+\cdots+2^ma_{2^m}
\end{align*}
When $2^m\leq n\leq 2^{m+1}$, we have
\begin{align*}
  S_n&\leq a_1+(a_2+a_3)+\cdots+(a_{2^m}+\cdots+a_{2^{m+1}-1})\\ 
  &\leq a_1+2a_2+\cdots+2^ma^{2^m}=T_m
\end{align*}
and 
\begin{align*}
  S_n&\geq a_1+a_2+(a_3+a_4)+\cdots+(a_{2^{m-1}+1}+\cdots+a_{2^m})\\
  &\geq \frac12 a_1+a_2+2a_4+\cdots+2^{m-1}a_{2^m}=\frac12T_m
\end{align*}
Thus, we have $S_n$ is bounded above $\iff$ $T_m$ is abounded above. This completes the proof.
\end{theorem}

\begin{example}
  $\Sigma \frac1{n^p}$ converges if $p>1$ and diverges if $p\leq 1$.

\textit{Proof:} If $p\leq 0$, divergence follows from the vanishing condition of series. \\
If $p>0$, \textit{Theorem \thesubsection.2} is applicable, we are led to the series 
\[\sum\limits_{k=0}^{\infty}2^k\cdot\frac1{2^{kp}}=\sum\limits_{k=0}^{\infty}2^{(1-p)k}=\sum\limits_{k=0}^{\infty}(2^{1-p})^k.\]
Now, $|2^{1-p}|<1\iff p>1$, and the proof follows by the first example of the previous subsection.
\end{example}

\begin{example}
  $\sum\limits_{n=2}^{\infty} \frac1{n(\ln n)^p}$ converges if $p>1$ and diverges if $p\leq 1$.

\textit{Proof:} We consider the series 
\[\sum\limits_{k=1}^{\infty}2^k\cdot\frac1{2^k(\ln 2^k)^p}=(\ln 2)^{-p}\sum\limits_{k=1}^\infty\frac1{k^p}\]
By \textit{Example \thesubsection.2}, we can easily prove the original proposition.
\end{example}

\begin{corollary}
  The above procedure may evidently be continued. In other words, we can also conclude 
  \[\sum\limits_{n=3}^\infty\frac1{n\ln n(\ln(\ln n))^p}\] converges if and only if $p>1$, and so forth. For instance,
  \begin{align}
    \sum\limits_{n=3}^\infty\frac1{n\ln n(\ln(\ln n))} 
  \end{align}
  diverges, whreas
  \begin{align}
    \sum\limits_{n=3}^\infty\frac1{n\ln n(\ln(\ln n))^2}
  \end{align}
  converges.
\end{corollary}

\begin{introduction}
  \hspace*{2em}
  From the above corollary, we may now observe that the series $(\thesection.2)$ differ very little from those of $(\thesection.3)$. Still, one diverges, and the other converges. If we continue the process which led us from \textit{Example \thesubsection.1} to \textit{Example \thesubsection.2}, and then to $(\thesection.2)$ and $(\thesection.3)$, we also get pairs of convergent and divergent series whose terms differ even less than those of $(\thesection.2)$ and $(\thesection.3)$.\\
  \hspace*{2em}
  Thus, one might be led to the conjecture that there is a limiting situation of some sort, i.e. a "boundary" with all convergent series on one side, all divergent series on the other side.
  Or when n is sufficiently large, Is there a largest convergent series?\\
  \hspace*{2em}
  In the following, We shall show that this conjecture is false.
\end{introduction}

\begin{proposition}
  Suppose positive seires $\Sigma a_n$ converges, then there must exist positive series $\Sigma b_n$ converges and satisfies $\lim\limits_{n\to\infty}\frac{b_n}{a_n}=+\infty$.

\textit{Proof:} Let 
\[S_n=a_1+\cdots+a_n,\]
\[T_n=a_n+a_{n+1}+\cdots=\sum\limits_{k=n}^\infty a_k.\]
We initially prove that $S_n$ converges $\iff$ $T_n$ converges to $0$. We can easily prove that $T_n$ converges iff $T_n$ converges to $0$. Hence, next we use the Cauchy criterion to prove
\[S_n ~converges \iff T_n~ converges.\]
Since for $m>n$,
\[T_m-T_n=a_n+\cdots+a_{m-1}=S_{m-1}-S_{n-1}\]
So we have for every $\varepsilon>0$, there exists $N_1>0$, such that for $m>n>N$, $|T_m-T_n|<\varepsilon$, if and only if for every $\varepsilon>0$, there exists $N_1>0$, such that for $m>n>N$, $|S_m-S_n|<\varepsilon$.
In other words, the above proposition holds.\\
By
\[a_n=T_n-T_{n+1}=(\sqrt{T_n}-\sqrt{T_{n+1}})(\sqrt{T_n}+\sqrt{T_{n+1}}),\]
we set $b_n=\sqrt{T_n}-\sqrt{T_{n+1}}$, and $B_n=\Sigma b_n=\sqrt{T_1}-\sqrt{T_n+1}$.\\
Hence, $\Sigma b_n$ converges owing to the convergence of $T_n$, and we have 
\[\frac{b_n}{a_n}=\frac1{\sqrt{T_n}+\sqrt{T_{n+1}}}\to \infty,\quad (n\to \infty)\]
\end{proposition}

\introduction
\hspace*{2em}
From the proof of \textit{Theorem \thesubsection.2}, we can know that for the non-negetive series $\Sigma a_n, \Sigma b_n$, if $A_n\leq B_n$ and $\Sigma b_n$ converges, then $\Sigma a_n$ converges.\\
\hspace*{2em}
This is rooted from the \textit{Theorem \thesubsection.1}, we can expand this to more general case.

\begin{theorem}[Comparison Test]
  \begin{enumerate}
    \item If $|a_n|\leq c_n$ for $n\geq N$, where $N$ is siome fixed integer, and if $\Sigma c_n$ converges , then $\Sigma a_n$ converges.
    \item If $a_n \geq d_n\geq 0$ or $a_n \leq d_n\leq0$ for $n\geq N$, and if $\Sigma d_n$ diverges ,then $\Sigma a_n$ diverges.
  \end{enumerate}
\end{theorem}

\begin{remark}
  It is also possible to multiply $c_n$ and $d_n$ by a constant $C>0$.
\end{remark}

\begin{remark}
  Whether analyzing a single series or comparing two series, we actually don't need to consider the finite initial terms. That is to say, we can only consider the case when n is sufficiently large, and we can even remove the finite terms from one series to compare it with another series.
\end{remark}

\begin{remark}
  Maybe someone would ask how to find the appropriate $N$. In fact, sometimes, when we utilize the functions to compare the sequences, we may find the inequalities hold only if $n$ is sufficiently large. For instance, for the series $\Sigma \frac{e^n}{n^{\ln n}}$, we compare it with $Sigma 1$, and we can easily find that $e^x>n\ln n$ holds indeed when $n$ is sufficiently large. Thus the original seires diverges.
\end{remark}

\subsection{The Root and Ratio Tests}
\begin{introduction}
  \hspace*{2em}
  The above comparison test could generate other useful tests if we let $b$ be some series whose convergence or divergence is known.\\
  \hspace*{2em} 
  Sometimes if we cannot find some series which have some relationship with the original series, we can try some typical series like $\Sigma q^n $.
\end{introduction}

\begin{theorem}[Root Test]
  Given $\Sigma a_n$, put $\alpha=\overline{\lim\limits_{n\to\infty}}\sqrt[n]{|a_n|}>0$.\\
  Then 
  \begin{itemize}
    \item if $\alpha<1$, $\Sigma a_n$ converges;
    \item if $\alpha >1$, $\Sigma a_n$ diverges;
    \item if $\alpha=1$, the test gives no information.
  \end{itemize}
  \textit{Proof:} By
  \[\alpha=\overline{\lim\limits_{n\to\infty}}\sqrt[n]{|a_n|},\]
  we can know
  \[
    \forall\, \varepsilon>0,~\exists\,N>0,~s.t.~\forall\,n>N,~|\sup \{\sqrt[k]{|a_k|}\mid k\geq n\}-\alpha|<\varepsilon,
  \]
  i.e.
  \[\alpha-\varepsilon<\sup \{\sqrt[k]{|a_k|}<\alpha+\varepsilon,\]
  if $\alpha<1$, it follows that
  \[\forall\,k\geq n\geq N,~ \sqrt[k]{|a_k|}<\alpha+\varepsilon,\]
  Because $0<\alpha<1$, we can set $\varepsilon$ such that $\alpha+\varepsilon=q<1$,\\
  i.e. \[\forall\,k\geq n\geq N,~ |a_k|<(\alpha+\varepsilon)^k=q^k,~(0<q<1).\]
  Since $\Sigma q^n$ converges if $|q|<1$, $\Sigma |a_n|$ also converges.\\
  Then it follows that $\Sigma a_n$ converges by absolute value proposition of series.\\
  If $\alpha>1$, it follows that
  \[\forall\,n\geq N,~\exists\,k\geq n,~s.t.~ \sqrt[k]{|a_k|}>\alpha-\varepsilon,\]
  Because $\alpha>1$, we can set $\varepsilon$ such that $\alpha-\varepsilon=q>1$,\\
  i.e. \[\forall N>0,~\exists\,k\geq N,~ |a_k|>(\alpha-\varepsilon)^k=q^k>1.\]
  Thus, we can conclude that $|a_n|$ fails to coverge to $0$, and then it follows that $a_n$ also fails to converge to $0$.\\
  By vanishing condition, $\Sigma a_n$ diverges.\\
  To prove the case $\alpha=1$, we need to consider the series
  \[\Sigma\frac1n,\quad\Sigma\frac1{n^2}.\]
  For each of these seires $\alpha=1$, but the first diverges and the seconde converges.
\end{theorem}
  
\begin{remark}
  The above theorem also applies to the case where the limit exists, and we can also write it in the form for sufficiently large $n$.
\end{remark}

\begin{introduction}
\hspace*{2em} when we utilize the comparison test, we just want to find a constant $C>0$ and a positive series $\Sigma b_n$ such that $\frac{|a_n|}{b_n}<C$ when $n$ is sufficiently large. And we can find that if when $n$ is sufficiently large, $\{\frac{|a_n|}{b_n}\}$ is monotonically decreasing, then the previous equation holds.\\
\hspace*{2em} In other words, $|\frac{a_{n+1}}{a_n}|\leq \frac{b_{n+1}}{b_n}$ for $n>N$ where $N$ is a fixed integer. And next we can let $b_n$ be some typical series like $\Sigma q^n$.
\end{introduction}

\begin{theorem}[Ratio Test]
The seires $\Sigma a_n$
\begin{itemize}
  \item converges if $\limsup\limits_{n\to\infty}|\frac{a_{n+1}}{a_n}|<1$,
  \item diverges if $\liminf\limits_{n\to\infty}|\frac{a_{n+1}}{a_n}|>1$.
  \item cannot be determined to be convergent or divergent in other cases.
\end{itemize}
\textit{Proof:} If $\limsup\limits_{n\to\infty}|\frac{a_{n+1}}{a_n}|<1$, we can find $0<q<1$, and an interger $N_1$, such that 
\[|\frac{a_{n+1}}{a_n}|<q=\frac{q^{n+1}}{q^n}\]
for $n>N_1$, so it follows that $\{\frac{|a_n|}{q^n}\}$ is monotonically decreasing when $n$ is sufficiently large. Then the original proposition holds.\\
If $\liminf\limits_{n\to\infty}|\frac{a_{n+1}}{a_n}|<1$, we can find $r>1$, and an interger $N_2$, such that 
\[|\frac{a_{n+1}}{a_n}|>r>1\]
for $n>N_2$, and it is easily seen that the vanishing condition $a_n\to0$ does not hold and the second proposition follows.

\end{theorem}

\begin{remark}
  The above two theorems can also be written into the form of sufficient $n$. And this form can usually include the case when the ratio equal $1$. However, in the next, we will introduce a more refined test to tackle the case when the previous theorem can not determine, i.e. the ratio $\frac{a_n}{a_{n+1}}$ is close to 1, but may fluctuate slightly around 1 or have higher-order corrections although $n$ is sufficiently large.
\end{remark}

\begin{theorem}[Guass Test]
  Given a positive series $\Sigma a_n$, Suppose that for sufficiently large $n$, the ratio of successive terms admits an asymptotic expansion:
  \[\frac{a_n}{a_{n+1}}=1+\frac hn+\frac{p_n}{n^\alpha},\]
  where $\alpha>1$ and $p_n$ is bounded, then $\Sigma a_n$ satisfies:
  \begin{itemize}
    \item if $h>1$, $\Sigma a_n$ converges.
    \item if $h<1$, $\Sigma a_n$ diverges.
    \item if $h=1$, $\Sigma a_n$ is inconclusive.
  \end{itemize}
\end{theorem}

\begin{corollary}
the above theorem have another form: Suppose that for sufficiently large $n$, the ratio of successive terms admits an expansion:
  \[\frac{a_n}{a_{n+1}}=1+\frac hn+o(\frac1{n\ln n}),\]
  then $\Sigma a_n$ satisfies:
  \begin{itemize}
    \item if $h>1$, $\Sigma a_n$ converges.
    \item if $h\leq1$, $\Sigma a_n$ diverges.
  \end{itemize}
\end{corollary}

\begin{theorem}[Raabe Test]
  If we turn the Guass test into the limit form, then we can obtain the Rabbe test: Given a positive series $\Sigma a_n$, set
  \[\lim\limits_{n\to\infty} n(\frac{a_n}{a_{n+1}}-1)=\alpha,\]
  \begin{itemize}
    \item if $\alpha>1$, $\Sigma a_n$ converges;
    \item if $\alpha<1$, $\Sigma a_n$ diverges;
    \item if $\alpha=1$,the test gives no informaiton.
  \end{itemize}
\end{theorem}

\begin{remark}
  The Raabe test and Guass test is only applicable for the positive series, but the first two can be used for all series.
\end{remark}

\begin{remark}
  When we reseach the convergence or divergence of series $\Sigma a_n$, we can all initially determine $\Sigma |a_n|$ which is a non-positive series and even a positive series. $\Sigma |a_n|$ is more easily than $\Sigma a_n$ to analysis and can be applied by more tests. In the above root or ratio tests, actually, we all urilize this thought, just first analysis $\Sigma |a_n|$ and then think how to connect the original seires $\Sigma a_n$. 
\end{remark}

\begin{remark}
  During the solution process, if the terms involve power functions or expressions with $n$-th powers, the root test is usually applied. If factorials or exponential terms appear, the ratio test is typically used. When the ratio test gives a limit equal to one, Raabe test is employed. If Raabe test is still inconclusive, one then turns to Gauss test or its corollaries, often with the aid of Taylor expansions for further analysis.
\end{remark}




\end{document}
